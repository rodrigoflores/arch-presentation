\documentclass{beamer}
\usepackage[brazil]{babel}
%\usepackage[latin1]{inputenc}
\usepackage[utf8x]{inputenc} 
%\usepackage[all]{xy}

\usetheme{Darmstadt}

\title{Introdução ao Archlinux\\I Encontro de Usuários Archlinux\\Fórum Internacional do Software Livre $10$}

\author{Rodrigo L. M. Flores \\ \url{flores@archlinux-br.org}}

\institute{Projeto Archlinux-BR}

\begin{document}

\date{\today}

\frame{\titlepage}
\section{I Encontro nacional de Usuários Archlinux}

%Bem-vindos

\begin{frame}
    \frametitle{Bem vindos}
    \begin{block}{Bem vindos ao I Encontro de Usuários Archlinux-br}
        Palestras que teremos:
        \begin{itemize}
            \item<1-> Introdução ao Arch Linux por Rodrigo Flores
            \item<2-> Migrando para o Arch Linux por Kessia ``even'' Pinheiro
            \item<3-> Arch Linux Brasil: o que fomos, somos e seremos por Douglas Soares de Andrade
            \item<4-> Construindo pacotes para o Arch Linux por Hugo Dória
        \end{itemize}
    \end{block}
\end{frame}

%Introdução 

\begin{frame}
    \frametitle{Introdução}
    \begin{block}{Mas o que é o Archlinux?}
        Uma distribuição Linux otimizada feita para ser leve e simples.
    \end{block}

\end{frame}

%Minimalismo 

%Falar de roupas

%No que o Arch foi baseado

%Como saber se o Arch é para você

%Inicialização BSD ? O que é isso?

%Arch e Ubuntu

%Arch e Debian

%Arch e Slackware

%Arch e Gentoo

%Depoimentos

%Dúvidas


\end{document}


% vim:set ts=4 expandtab:
