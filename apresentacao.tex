\documentclass{beamer}
\usepackage[brazil]{babel}
%\usepackage[latin1]{inputenc}
\usepackage[utf8x]{inputenc} 
%\usepackage[all]{xy}

\usetheme{Darmstadt}

\title{Introdução ao Archlinux\\I Encontro de Usuários Archlinux\\Fórum Internacional do Software Livre $10$}

\author{Rodrigo L. M. Flores \\ \url{flores@archlinux-br.org}}

\institute{Projeto Archlinux-BR}

\begin{document}

\date{\today}

\frame{\titlepage}
\section{I Encontro nacional de Usuários Archlinux}

%Bem-vindos

\begin{frame}
    \frametitle{Bem vindos}
    \begin{block}{Bem vindos ao I Encontro de Usuários Archlinux-br}
        Palestras que teremos:
        \begin{itemize}
            \item<1-> Introdução ao Arch Linux por Rodrigo Flores
            \item<2-> Migrando para o Arch Linux por Kessia ``even'' Pinheiro
            \item<3-> Arch Linux Brasil: o que fomos, somos e seremos por Douglas Soares de Andrade
            \item<4-> Construindo pacotes para o Arch Linux por Hugo Dória
        \end{itemize}
    \end{block}
\end{frame}

\section{Introdução}

%Introdução 

\begin{frame}
    \frametitle{Introdução}
    \begin{block}{Pergunta}<1->
        O que é o Archlinux ?
    \end{block}
    \begin{block}{Resposta}<2->
            O Archlinux é uma distribuição Linux feita para ser ``peso \textbf{leve}'' e \textbf{simples}. Possui suporte para as arquiteturas i686 e x86-64. 
    \end{block}
    
\end{frame}

\begin{frame}
    \frametitle{i686 e x86-64?}
    \begin{block}{i686?}<1->
        Na prática, qualquer PC baseado em Intel com processador ``melhor'' que Pentium Pro (lançado em 1995) é da arquitetura i686. 
    \end{block}
    \begin{block}{x86-64}<2->
        Processadores $64$ bits (AMD $64$, acima de Pentium D). 
    \end{block}
    \begin{block}{E o que o Arch tem de bom?}<3->
        Na prática, PCs piores que Pentium Pro hoje são sucata. Então o Arch é uma distribuição que é compatível com praticamente qualquer PC existente.
    \end{block}

\end{frame}


\begin{frame}
    \frametitle{Peso Leve}
    \begin{block}{O que peso leve quer dizer ?}
        Em inglês ``lightweight''. Por padrão o Arch instala só o Kernel do Linux e mais alguns pacotes essenciais (grub ou lilo, utilitários de rede e etc).
    \end{block}

    \begin{block}{Vantagens disso?}
        Te permite ao máximo personalizar teu sistema. Permite instalar teu sistema em um Pentium Pro. 
    \end{block}
\end{frame}

%Falar de roupas

\begin{frame}
    \frametitle{Analogia com roupas}
    \begin{block}{Pergunta}
            Qual a roupa que todas as pessoas usam ?

    \end{block}

    %Colocar foto de uma cueca
\end{frame}


%Como saber se o Arch é para você

%Inicialização BSD ? O que é isso?

%Depoimentos

%Dúvidas


\end{document}


% vim:set ts=4 expandtab:
