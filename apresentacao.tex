\documentclass{beamer}
\usepackage[brazil]{babel}
%\usepackage[latin1]{inputenc}
\usepackage[utf8x]{inputenc} 
%\usepackage[all]{xy}

\usetheme{Darmstadt}

\title{Introdução ao Archlinux\\I Encontro de Usuários Archlinux\\Fórum Internacional do Software Livre $10$}

\author{Rodrigo L. M. Flores \\ \url{flores@archlinux-br.org}}

\institute{Projeto Archlinux-BR}

\begin{document}

\date{\today}

\frame{\titlepage}
\section{I Encontro nacional de Usuários Archlinux}

%Bem-vindos

\begin{frame}
    \frametitle{Bem vindos}
    \begin{block}{Bem vindos ao I Encontro de Usuários Archlinux-br}
        Palestras que teremos:
        \begin{itemize}
                \item<1-> \textbf{Introdução ao Arch Linux por Rodrigo Flores}
            \item<2-> Migrando para o Arch Linux por Kessia ``even'' Pinheiro
            \item<3-> Arch Linux Brasil: o que fomos, somos e seremos por Douglas Soares de Andrade
            \item<4-> Construindo pacotes para o Arch Linux por Hugo Dória
        \end{itemize}
    \end{block}
\end{frame}

\section{Introdução}


\begin{frame}[fragile]
    \frametitle{Quem sou eu ?}
    \begin{block}{Quem sou eu ?}
        \begin{itemize}
            \item<1-> Estudante de ciência da computação do IME - USP 
            \item<2-> Tradutor de SL (GNOME, Pidgin, Mono)
            \item<3-> Desenvolvedor e membro da comissão de coordenação do Archlinux-br
            \item<4-> Sósia do Linus Torvalds \verb#:-P# 
        \end{itemize}
        %Colocar foto de mim e do Linus
    \end{block}
    \begin{figure}[ht]
        \includegraphics<4->[scale=0.3]{twins.jpg}
    \end{figure}


\end{frame}


%Introdução 

\begin{frame}
    \frametitle{Introdução}
    \begin{block}{Pergunta}<1->
        O que é o Archlinux ?
    \end{block}
    \begin{block}{Resposta}<2->
            O Archlinux é uma distribuição Linux feita para ser ``\textbf{peso leve}'' e \textbf{simples}. Possui suporte para as arquiteturas i686 e x86-64. 
    \end{block}
    
\end{frame}

\begin{frame}
    \frametitle{i686 e x86-64?}
    \begin{block}{i686?}<1->
        Na prática, qualquer PC baseado em Intel com processador ``melhor'' que Pentium Pro (lançado em 1995) é da arquitetura i686. 
    \end{block}
    \begin{block}{x86-64}<2->
        Processadores $64$ bits (AMD $64$, acima de Pentium D). 
    \end{block}
    \begin{block}{E o que o Arch tem de bom?}<3->
        Na prática, PCs piores que Pentium Pro hoje são sucata. Então o Arch é uma distribuição que é compatível com praticamente qualquer PC existente.
    \end{block}

\end{frame}

\section{À la Arch}

\begin{frame}
    \frametitle{Simples}
    \begin{block}{Simplicidade?}
        O Archlinux define simplicidade como uma base UNIX-like, sem adições,
        desnecessárias, modificações ou complicações, que permitem a um usuário modelar o sistema de acordo com suas 
        necessidades. Resumindo: uma elegante abordagem minimalista.
    \end{block}
\end{frame}


\begin{frame}
    \frametitle{Código funcionando ao invés de conveniência} 
    \begin{block}{Pacotes intactos}
        Patches nos pacotes quase nunca. Quando tem, é para o pacote funcionar ou por problemas de licença (firefox). Desenvolvedores do 
        arch não adicionam features em softwares que não são do próprio arch, nem tentam deixar o software mais amigável.
    \end{block}
\end{frame}

\begin{frame}
    \frametitle{Centrada no usuário e ``livre''}
    \begin{block}<1->{Quem manda aqui é você!}
        O Arch dá aos usuários controle completo sobre o sistema. O Arch te dá a liberdade de fazer o que quiser.
    \end{block}
    %Colocar foto do Chuck Norris
    \begin{block}{Lembre-se da frase do tio do Peter Parker}
        Com o poder grande, vem junto a grande responsabilidade.
    \end{block}
\end{frame}



\section{Principais ferramentas}

\begin{frame}[fragile]
        \frametitle{Principais ferramentas e coisas legais}
        \begin{block}{O que é ?}
            \begin{itemize}
                \centering
                \item[Pacman] Gerenciador de pacotes
                \item[Makepkg] Ferramenta de compilação de pacotes, usando uma ``receita de bolo''
                \item[AUR] Repositório de usuários, onde você pode baixar as ``receitas de bolo''
                \item[ABS] ``Receitas de bolo'' de compilação dos pacotes do repositório
            \end{itemize}
        \end{block}
\end{frame}

\section{Como fazer X ? }

\begin{frame}
    \frametitle{Comofas/}
    \begin{block}{Lugares para pedir ajuda...}
        \begin{itemize}
            \centering
            \item[Wiki] \url{http://wiki.archlinux.org}. 
            \item[Wiki-br] \url{http://wiki.archlinux-br.org}. 
            \item[Forum] \url{http://bbs.archlinux.org/}  
            \item[Forum-br] \url{http://forum.archlinux-br.org}%Colocar endereço 
            \item[Lista] \url{http://www.archlinux.org/mailman/listinfo/arch-general}
            \item[Lista-br] \url{http://groups.google.com/group/archlinux-br/}
            \item[Bug report] \url{http://bugs.archlinux.org/} 
        \end{itemize}
    \end{block}        

\end{frame}


\section{Quero ser um archer }
    
\begin{frame}
    \frametitle{O que preciso saber para ser um Archer?}
    \begin{block}{Preciso ser um hacker ?}
        \begin{itemize}
            \item<1-> Se virar em um terminal. 
            \item<2-> Saber usar algum editor de texto em modo texto (vim, emacs, nano) 
            \item<3-> Saber configurar a internet em modo texto (ou ter um PC ativo enquanto instala)
            \item<4-> Saber buscas as coisas na internet
            \item<5-> Saber o princípio ``Procurar e depois perguntar''
        \end{itemize}
    \end{block}
\end{frame}

\begin{frame}
        \frametitle{Depoimentos de novos Archers}
        \begin{block}{Bons depoimentos de novos usuários :}
            \begin{itemize}
                \item \url{http://jfmitre.com/2009/06/mudando-radicalmente-distribuicao.html} por J. F. Mitre
                \item por Thadeu Penna
            \end{itemize}
        \end{block}
\end{frame}


\section{Dúvidas?}

\begin{frame}
    \frametitle{Dúvidas}
    \begin{block}{Contato :}
        \begin{itemize}
            \centering
            \item[E-mail] flores@archlinux-br.org        
            \item[XMPP]  im@rodrigoflores.org        
            \item[Site]  \url{http://rodrigoflores.org}
            \item[Site do arch-br]  \url{http://flores.archlinux-br.org}
            \item[Blog]  \url{http://blog.rodrigoflores.org}        
            \item[Twitter] rodrigoflores        
            \item[Identi.ca] rodrigoflores        
            \item[Jaiku] flores        
        \end{itemize}
    \end{block}

\end{frame}

\begin{frame}
    \frametitle{Ganhe uma mídia do Arch}
    \begin{block}{Ganhe uma mídia do Archlinux}
            Os três primeiros tweets (ou identi.cas ou jaikus) pra mim dizendo 
            ``Quero experimentar o \#archlinux \#fisl10'', ganham uma mídia da distribuição
    \end{block}
\end{frame}


\end{document}


% vim:set ts=4 expandtab:
